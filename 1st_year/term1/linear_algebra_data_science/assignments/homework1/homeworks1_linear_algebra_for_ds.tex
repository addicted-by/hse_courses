\documentclass[12pt]{report}
\usepackage{../mystyle}
\begin{document}
\boldmath
\fancyhead[R]{Homework \thesection \mygreen{(Optional)}}
\fancyhead[C]{Linear Algebra for Data Science}

\setcounter{section}{1}

\begin{problem}{Pseudoinverse Matrix}
    % \begin{problem}[Теорема Бэра]
    Find 
    \[
    \begin{pmatrix}
      0\\
      1\\
      2\\
      3
    \end{pmatrix}^{+}
    \]
    \end{problem}
    
    \begin{solution}
    From the lecture we know that $A^{+} = (A^{*}A)^{-1}A^{*}$.
    
    \begin{gather*} 
        A^{*} = A^T = 
    \begin{pmatrix}
      0 & 1 & 2 & 3
    \end{pmatrix}
    ;\ A^{*}A = 
    \begin{pmatrix}
      0 & 1 & 2 & 3
    \end{pmatrix}
    \begin{pmatrix}
      0\\
      1\\
      2\\
      3
    \end{pmatrix}
     = 0 + 1 + 4 + 9 = 14\\ (A^{*}A)^{-1} = \medmath{\frac{1}{14}} \Rightarrow A^{+} = (A^{*}A)^{-1}A^{*} = \medmath{\frac{1}{14}}
     \begin{pmatrix}
      0 & 1 & 2 & 3
    \end{pmatrix}. \end{gather*}
    \end{solution}
    
    \vspace{\baselineskip}
    
    \begin{problem}{Skeletonization}
        Find a full rank decomposition
        \[
    \begin{pmatrix}
        1 & 1 & 1\\
        2 & 2 & 2\\
        3 & 3 & 3\\
        1 & 2 & 3
    \end{pmatrix}
        \]
    \end{problem}
    
    \begin{solution}
        We want such decomposition: $A_{m \times n} = F_{m \times r} \times G_{r \times n}$; m = 4, n = 3. Use Gauss transformation to find G:
        
    $\begin{pmatrix}
        1 & 1 & 1\\
        2 & 2 & 2\\
        3 & 3 & 3\\
        1 & 2 & 3
    \end{pmatrix}
    \sim
    \begin{pmatrix}
        1 & 1 & 1\\
        1 & 2 & 3\\
        0 & 0 & 0\\
        0 & 0 & 0
    \end{pmatrix}
    \sim
    \begin{pmatrix}
        1 & 1 & 1\\
        0 & 1 & 2\\
        0 & 0 & 0\\
        0 & 0 & 0
    \end{pmatrix}
    \sim
    \begin{NiceArray}{(ccc)}[cell-space-top-limit=3pt]
        1 & 0 & -1\\
        0 & 1 & 2\\
        0 & 0 & 0\\
        0 & 0 & 0
    \CodeAfter
      \tikz \draw (2-|1) -| (3-|2) -| (3-|4) ; 
    \end{NiceArray}$; So, $G = 
    \begin{pmatrix}
        1 & 0 & -1\\
        0 & 1 & 2
    \end{pmatrix}$
    
    We see that first and second columns $(a_1, a_2)$ are independent and $a_3 = -a_1 + 2a_2$, so matrix F will consist of $a_1$ and $a_2$:
    
    $F = \begin{pmatrix}
        1 & 1\\
        2 & 2\\
        3 & 3\\
        1 & 2
    \end{pmatrix}$. Let's check: $A = F \times G = 
    \begin{pmatrix}
        1 & 1\\
        2 & 2\\
        3 & 3\\
        1 & 2
    \end{pmatrix}
    \begin{pmatrix}
        1 & 0 & -1\\
        0 & 1 & 2
    \end{pmatrix} = 
    \begin{pmatrix}
        1 & 1 & 1\\
        2 & 2 & 2\\
        3 & 3 & 3\\
        1 & 2 & 3
    \end{pmatrix}
    $.
    
    
    \end{solution}
    \begin{problem}{Pseudoinverse matrix. Skeletonization}
        Find 
        \[
    \begin{pmatrix}
        1 & 0 & 0\\
        1 & 1 & 1\\
        0 & 1 & 1
    \end{pmatrix}^{+}
        \]
    \end{problem}
    
    \begin{solution}
        Check $\rank(A)$: 
        $\begin{pmatrix}
            1 & 0 & 0\\
        1 & 1 & 1\\
        0 & 1 & 1
        \end{pmatrix}
        \sim
        \begin{NiceArray}{(ccc)}[cell-space-top-limit=3pt]
        1 & 0 & 0\\
        0 & 1 & 1
    \CodeAfter
      \tikz \draw (2-|1) -| (3-|2) -| (3-|3); 
    \end{NiceArray} = G
    \Rightarrow \rank(A) < \dim(A) \Rightarrow$
    we need to use formula 
    \[A^{+} = G^{+}F^{+} = G^{*}(GG^{*})^{-1}(F^{*}F)^{-1}F^{*}\]
    \begin{gather*}
        G^{*} = G^T = 
        \begin{pmatrix}
            1 & 0\\
            0 & 1\\
            0 & 1
        \end{pmatrix}; GG^{*} = 
        \begin{pmatrix}
            1 & 0\\
            0 & 2
        \end{pmatrix}; (GG^{*})^{-1} = 
        \begin{pmatrix}
            1 & 0 & \vrule & 1 & 0 \\
            0 & 2 & \vrule & 0 & 1
        \end{pmatrix}
        \sim
        \begin{pmatrix}
            1 & 0 & \vrule & 1 & 0 \\
            0 & 1 & \vrule & 0 & \medmath{\frac{1}{2}}
        \end{pmatrix}
        \Longrightarrow 
        \begin{pmatrix}
            1 & 0\\
            0 & \medmath{\frac{1}{2}}
        \end{pmatrix}, \\ G^{*}(GG{*})^{-1} = 
        \begin{pmatrix}
            1 & 0\\
            0 & \medmath{\frac{1}{2}}\\[0.2cm]
            0 & \medmath{\frac{1}{2}}
        \end{pmatrix}
\end{gather*}

    Then 
    \[ F = 
    \begin{pmatrix}
        a_1 & a_2
    \end{pmatrix}
     = 
     \begin{pmatrix}
         1 & 0\\
         1 & 1\\
         0 & 1
     \end{pmatrix},
     F^{*} = F^T = 
     \begin{pmatrix}
         1 & 1 & 0\\
         0 & 1 & 1
     \end{pmatrix}, \dots, (F^{*}F)^{-1}F{*} = \frac{1}{3}
     \begin{pmatrix}
         2 & 1 & -1\\
         -1 & 1 & 2
     \end{pmatrix}
     \]

    All things considered,
     \[ A^{+} = \frac{1}{3} \frac{1}{2}
     \begin{pmatrix}
        2 & 0\\
        0 & 1\\
        0 & 1
    \end{pmatrix}
     \begin{pmatrix}
         2 & 1 & -1\\
         -1 & 1 & 2
     \end{pmatrix} = \frac{1}{6} 
     \begin{pmatrix}
         4 & 2 & -2\\
         -1 & 1 & 2\\
         -1 & 1 & 2
     \end{pmatrix}
     \]
    
    To complete our solution we need to check all Moore-Penrose axioms, I leave it for the reader.
    \end{solution}
    
    \begin{problem}{Proof}
        Prove that $Im(AA^{+}) = Im(AA^{*}) = Im(A)$.
    \end{problem}
    
    \begin{solution}
        to do
    \end{solution}

\end{document}
