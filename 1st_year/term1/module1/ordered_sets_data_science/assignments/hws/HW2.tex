\documentclass[12pt]{report}
\usepackage{../mystyle}
\begin{document}
\boldmath
\fancyhead[L]{Homework 2.}
\fancyhead[C]{Ordered sets in Data Analysis}
\fancyhead[R]{Ryabykin Aleksey}
\begin{problem}{}
    What is the worst-case complexity of computing the inverse relation in case where the relation is represented 
    \begin{itemize}
        \item[a) ] by the set of pairs of the relation;
        \item[b) ] by the matrix of the relation?
    \end{itemize}
\end{problem}
\begin{solution}
Let $R \subseteq A \times A$ is the binary relation.
\begin{definition}{}{}
    $R^d := \{(a,b)\ | \ (b,a) \in R\}$
\end{definition}
\begin{itemize}
    \item[a) ] Let $S$ is the set of pairs for relation $R$. Then to compute inverse relation we need to iterate through every pair in the set $S$ and push to the new set $S_1$ pairs from the first one set $S$ with swapped elements (for example, by the algo $std::swap$ complexity of  which one is $\bigOO(1)$). Then we need at least $\bigOO(n)$ operations, where $n$ is an amount of pairs in the set $S$.
    \item[b) ] Let's define the matrix of the relation by notation $M_{R}$. From the definition of the inverse relation we can obtain that our goal is to find transposed matrix to $M$. The worst-case time complexity to do it is $\bigOO(n\times m)$, where $n, m$ are the matrix shapes, because we need to iterate through matrix by two cycles.
\end{itemize}
\end{solution}

\begin{problem}{}
    What is the worst-case complexity of computing the product of relations in case where the relation is represented
    \begin{itemize}
        \item[a) ] by the set of pairs of the relation;
        \item[b) ] by the matrix of the relation?  
    \end{itemize}
\end{problem}
\begin{solution}
    \begin{definition}{}{}
        $P \cdot R = \left\{(x,y)| \ \exists z \ \ \left(\left(x, z\right) \in P \ \text{ and } \ (z,y) \in R\right)\right\},
        $ 
        where $P \subseteq A \times A, \ R \subseteq A \times A$ are two binary relations.
    \end{definition}
    \begin{itemize}
        \item[a) ] Let $S_1$ be the set of pairs for $P$ and $S_2$ same for $R$. To find the product of relations we need to iterate though all pairs $(a,b) \in S_1$ and for each such pair we need to iterate through all the pairs of a kind $(b,c) \in S_2$ and push it to final set with result of the product. For such operations we need $\bigOO(n \times m)$, where $n$ is the size of set $S_1$, $m$ is the size of set $S_2$
        \item[b) ]  Lets define $M_P$ and $M_R$ as a matrices for $P$ and $R$ relations. The result matrix element $i, j$ can be calculated by the formula:
        \[
            \bigcup\limits_{k=1}^{n} x_{ik} \land y_{kj}            
        \]
        We need to process all elements from $P[i,:]$ and $R[:, j]$. So there are $n^2$ elements in a result matrix $M$, so we have a worst-time complexity of $\bigOO(n^3)$. 
    \end{itemize}
\end{solution}

\begin{problem}{}
    What is the worst-case complexity of computing the transitive closure of a relation in case where the relation is represented
    \begin{itemize}
        \item[a) ] by the set of pairs of the relation;
        \item[b) ] by the matrix of the relation?  
    \end{itemize}
\end{problem}
\begin{solution}
    \begin{definition}{}{}
        \[
            R^T = R \cup R^2 \cup R^3 \cup \ldots = \bigcup\limits_{i=1}^\infty R^i,  
        \]
        where 
        \[
            R^n = \underbrace{R\times R\times \ldots \times R}_{n\ \text{ times }}  
        \]
    \end{definition}
    \begin{itemize}
        \item[a) ] We need to compute all the $R^i$ up to $n - 1$, where $n$ is the power of set of pairs, using algorithm from the previous problem. So in the worst case we have $\bigOO(n^3)$ complexity.
        \item[b) ] We can view on the matrix of the relation as an adjacency matrix of some graph $G$. Then the complexity of finding transitive closure will be the complexity for breadth-first search $\bigOO(V+E)$
    \end{itemize}
\end{solution}
\end{document}