% \begin{multicols}{2}
%     \raggedcolumns
    \section{Basic concepts and tasks of statistics. Types and sources of statistical data}
    Основные определения:
    \begin{itemize}
        \item Генеральная совокупность -- сведения о всех анализируемых объектах;
        \item Выборка -- множество результатов, отобранных из генеральной совокупности (репрезентативность);
        \item Объем совокупности -- число единиц, образующих совокупность;
        \item Неопределенность и вариация (основные характеристики статистики) -- при многократных измерениях происходят изменения;
        \item Признак -- характеристика единицы совокупности;
        \item Показатель (индикатор) -- количественная характеристика явления;
        \item Параметр -- относительно постоянная величина, характеризующая генеральную совокупность;
        \item Выборочная характеристика (статистика) -- эмпирический аналог параметра;
        \item Статистические выводы -- заключения, формируемые анализом эмпирических данных.
    \end{itemize}
    \subsection*{Виды и источники статистических данных}
    Статистические данные разделяются на:
    \begin{itemize}
        \item Пространственные: сведения об объектах наблюдения с различным порядком;
        \item Временные: хронологический порядок (моментные -- сумма только сумма значений и интервальные -- суммирование дает общую характеристику и может быть проинтерпретирована);
        \item Пространственно-временные: набор объектов в хронологическом порядке со сведениями об объектах. 
    \end{itemize}
    Статистические данные могут быть одномерными и многомерными, количественные и категориальные, первичные (регистрируемые для одного конкретного объекта) и агрегированные (объект -- совокупность других объектов). 
    \par 
    Шкалы измерения данных: качественные данные -- номинальная (профессия, пол, город), порядковая (место в рейтинге), количественные (непрерывные и дискретные) -- интервальная (температура воздуха), относительная (количество наличных денег, времени, объектов).
    \par 
    Источники статистических данных: непосредственные измерения, мнения экспертов, документированные значения.
    \begin{itemize}
        \item Статистическое наблюдение -- планомерный и систематический сбор данных об исследуемых явлениях и процессах, бывает сплошным (на генеральной совокупности) и несплошным (на выборке).
    \end{itemize}
    \subsection*{Задачи статистики}
    \underline{В узком смысле}: сжатие информации и наглядное представление результатов.
    \par
    \underline{В широком смысле}: обобщение результатов выборочного исследования на генеральную совокупность.
    \par
    Первичная обработка: пример данных \underline{качественного} характера
    \begin{itemize}
        \item таблица частот;
        \item столбиковая диаграмма;
        \item круговая диаграмма.
    \end{itemize}
    Первичная обработка: \underline{количественного} характера. \begin{itemize}
        \item гистограмма
    \end{itemize}
    \subsubsection*{Этапы статистического моделирования:}
    Определение цели и задач моделирования
    \begin{enumerate}
        \item Формализация -- преобразование объектов и отношений в математическую абстрактную модель;
        \item Сбор и квантификация данных -- предусматривает отражение данных в шкалах, их предварительная обработка -- избавление от ошибок;
        \item Спецификация модели -- представление в виде формул;
        \item Идентификация модели и ее анализ -- оценка параметров модели, ее характеристик;
    \end{enumerate}
    Верификация модели.
% \end{multicols}