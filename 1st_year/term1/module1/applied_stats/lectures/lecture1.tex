% \begin{multicols}{2}
%     \raggedcolumns
    \section{Basic concepts and tasks of statistics. Types and sources of statistical data}
    \subsection*{Основные определения:}
    \begin{itemize}
        \item Генеральная совокупность -- сведения о всех анализируемых объектах;
        \item Выборка -- множество результатов, отобранных из генеральной совокупности (репрезентативность);
        \item Объем совокупности -- число единиц, образующих совокупность;
        \item Неопределенность и вариация (основные характеристики статистики) -- при многократных измерениях происходят изменения;
        \item Признак -- характеристика единицы совокупности;
        \item Показатель (индикатор) -- количественная характеристика явления;
        \item Параметр -- относительно постоянная величина, характеризующая генеральную совокупность;
        \item Выборочная характеристика (статистика) -- эмпирический аналог параметра;
        \item Статистические выводы -- заключения, формируемые анализом эмпирических данных.
    \end{itemize}
    \subsection*{Виды и источники статистических данных}
    Статистические данные разделяются на:
    \begin{itemize}
        \item Пространственные: сведения об объектах наблюдения с различным порядком;
        \item Временные: хронологический порядок (моментные -- сумма только сумма значений и интервальные -- суммирование дает общую характеристику и может быть проинтерпретирована);
        \item Пространственно-временные: набор объектов в хронологическом порядке со сведениями об объектах. 
    \end{itemize}
    Статистические данные могут быть одномерными и многомерными, количественные и категориальные, первичные (регистрируемые для одного конкретного объекта) и агрегированные (объект -- совокупность других объектов). 
    \par 
    Шкалы измерения данных: качественные данные -- номинальная (профессия, пол, город), порядковая (место в рейтинге), количественные (непрерывные и дискретные) -- интервальная (температура воздуха), относительная (количество наличных денег, времени, объектов).
    \par 
    Источники статистических данных: непосредственные измерения, мнения экспертов, документированные значения.
    \begin{itemize}
        \item Статистическое наблюдение -- планомерный и систематический сбор данных об исследуемых явлениях и процессах, бывает сплошным (на генеральной совокупности) и несплошным (на выборке).
    \end{itemize}
    \subsection*{Задачи статистики}
    
    \par 
    \underline{В узком смысле}: сжатие информации и наглядное представление результатов.
    \par
    \underline{В широком смысле}: обобщение результатов выборочного исследования на генеральную совокупность.
    \par
    Первичная обработка: пример данных \underline{качественного} характера
    \begin{itemize}
        \item таблица частот;
        \item столбиковая диаграмма;
        \item круговая диаграмма.
    \end{itemize}
    \par
    Первичная обработка: \underline{количественного} характера. \begin{itemize}
        \item гистограмма
    \end{itemize}
    \subsubsection*{Этапы статистического моделирования:}
    Определение цели и задач моделирования
    \begin{enumerate}
        \item Формализация -- преобразование объектов и отношений в математическую абстрактную модель;
        \item Сбор и квантификация данных -- предусматривает отражение данных в шкалах, их предварительная обработка -- избавление от ошибок;
        \item Спецификация модели -- представление в виде формул;
        \item Идентификация модели и ее анализ -- оценка параметров модели, ее характеристик;
        \item Верификация модели.
    \end{enumerate}
\subsection*{Выборочные статистики}

\par
Выборочная статистика -- эмпирический аналог параметра. Выборочная характеристика является функцией от результатов наблюдений $\theta^* = \theta^* \left(x_1, \ldots, x_n\right)$

\subsubsection*{Порядковые статистики}
\begin{itemize}
    \item Медиана -- величина, разделяющая упорядоченный набор на 2 равные части -- $50\%$ всех наблюдений находятся ниже медианы, $50 \%$ выше.
    \item Первый квартиль -- величина, разделяющая упорядоченную выборку, $25\%$ всех наблюдений лежит ниже первого квартиля, $75\%$, соответственно, выше.
    \item Аналогично можно определить второй квартиль, причем, становится ясно, что понятия первого квартиля и медианы совпадают;
    \item Третий квартиль итеративно можно определить как разделяющую величину, ниже которой лежат $75\%$ наблюдений, оставшиеся $25\%$ выше.
    \item Первый дециль: $10\%$ наблюдений лежат ниже, $90\%$ выше.
    \item Интерквартильный размах $IQR$ -- разность между третьим и первым квартилями (служит мерой разброса)
\end{itemize}
\subsubsection*{Моментные характеристики положения и разброса}
\begin{itemize}
    \item Среднее значение -- сумма значений признака, деленная на число его значений. (характеристика положения);
    \item Дисперсия -- среднее значения квадрата отклонения результатов наблюдений от среднего значения (разброс);
    \item Среднее квадратическое отклонение -- положительный квадратный корень из дисперсии.
\end{itemize}
\subsection*{Важные статистики}
\begin{itemize}
    \item Выборочной характеристикой называется функция от результатов наблюдений:
    Можем определить выборочное среднее значение двумя способами:
    \[
        \begin{array}{c}
            \overline{x} = \dfrac{1}{n}\sum\limits_{i=1}^n x_i \hspace*{0.5cm} \text{среднее арифметическое}\\
            \overline{x} = \sqrt[n]{\prod\limits_{i=1}^n x_i} \hspace*{0.5cm} \text{среднее геометрическое}
        \end{array}  
    \]
    Так же можно определить выборочные дисперсию и среднеквадратичное отклонение:
    \[
        \begin{array}{c}
        \var (x_i) = \dfrac{1}{n}\sum\limits_{i=1}^n \left(x_i - \overline{x}\right)^2 \hspace*{0.5cm} \text{ выборочная дисперсия}\\
        S = \sqrt{\var(x_i)} \hspace*{0.5cm} \text{выборочное среднеквадратичное отклонение}  
        \end{array}
    \]
    Выборочные начальные и центральные моменты:
    \[
        \begin{array}{c}
            v^*_k = \dfrac{1}{n} \sum\limits_{i=1}^n x_i^k\\
            \mu_k^* = \dfrac{1}{n} \sum\limits_{i=1}^n \left(x_i - \overline{x}\right)^k
        \end{array}  
    \]
    Коэффициенты асимметрии и эксцесса:
    \[
        \begin{array}{c}
        \beta_1 = \dfrac{\mu_3^*}{S^3} \\
        \beta_2 = \dfrac{\mu_4}{S^4} - 3 
        \end{array}       
    \]
\end{itemize}
\subsubsection*{Характеристики равномерности распределения количественного признака}
\begin{itemize}
    \item Кривая Лоренца;
    \item Коэффициент Джини:
    \[
        G = 1 - 2 \sum\limits_{i=1}^n x_i y_{i \text{ нак }} + \sum\limits_{i=1}^n x_iy_i.  
    \]
\end{itemize}