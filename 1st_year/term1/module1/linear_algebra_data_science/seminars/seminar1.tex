\section{Pseudoinverse matrices. Skeletonization. Singular value decomposition (SVD)}

\begin{problem}{Skeletonization. Pseudoinverse matrix}
    Find the pseudoinverse matrix to matrix $A$ using skeletonization:
    \[
             A = \begin{bmatrix}
                    2 & -1 & 0\\
                    -1 & 1 & 1\\
                    0 & 1 & 2
                \end{bmatrix}
    \]
\end{problem}
\begin{solution}
    Let's start with skeletonization and then find the pseudoinverse matrix by the formula:
    \[
        A^+ = G^*(G,G^*)^{-1}(F^*,F)^{-1}F^* 
    \]
    \[
        A = \Vast[
            \overbrace{\fbox{$\begin{matrix}
                2 & -1 \\
                -1 & 1 \\
                0 & 1
            \end{matrix}$}}^{F} \hspace*{0.25cm} \begin{matrix}
                0 \\
                1 \\
                2
            \end{matrix}\Vast] \Rightarrow \begin{bmatrix}
                2 & -1 & 0 \\
                0 & \medmath{\frac{1}{2}} & 1 \\
                0 & 1 & 2
            \end{bmatrix} \Rightarrow \begin{bmatrix}
                1 & -\medmath{\frac{1}{2}} & 0\\
                0 & \medmath{\frac{1}{2}} & 1\\
                0 & 0 & 0
            \end{bmatrix} \Rightarrow \Vast[
                \begin{array}{c}
                    \overbrace{\fbox{$\begin{matrix}
                        1 & 0 & 1\\
                        0 & 1 & 2
                    \end{matrix}$}}^{G}\\
                    \hspace*{0.18cm} \begin{matrix}
                        0 & 0 & 0\\\\
                    \end{matrix}
                \end{array}
                \Vast]
    \]
    Let's check:
    \[
        F \cdot G = \begin{bmatrix}
            2 & -1 \\
            -1 & 1 \\
            0 & 1
        \end{bmatrix} \cdot \begin{bmatrix}
            1 & 0 & 1 \\
            0 & 1 & 2
        \end{bmatrix} = \begin{bmatrix}
            2 & -1 & 0\\
            -1 & 1 & 1\\
            0 & 1 & 2
        \end{bmatrix}
    \]
    Correct. Now we need to find the pseudoinverse matrix:
    \[
        G^*(G, G^*)^{-1} = \begin{bmatrix}
            1 & 0 \\
            0 & 1 \\
            1 & 2
        \end{bmatrix} \cdot \left(\begin{bmatrix}
            1 & 0 & 1\\
            0 & 1 & 2
        \end{bmatrix} \cdot \begin{bmatrix}
            1 & 0 \\
            0 & 1\\
            1 & 2
        \end{bmatrix}\right)^{-1} = \begin{bmatrix}
            1 & 0\\
            0 & 1\\
            1 & 2
        \end{bmatrix} \cdot \dfrac{1}{6} \begin{bmatrix}
            5 & -2 \\
            -2 & 2
        \end{bmatrix} = \begin{bmatrix}
            5 & -2 \\
            -2 & 2\\
            1 & 2
        \end{bmatrix} \cdot \dfrac{1}{6}
    \]
    \begin{note}{}{}
        Matrix $(G, G^*)$ is called Gramm Matrix and contains results of scalar products. 
    \end{note}
    \begin{gather*}
        (F^*, F)^{-1}  F^* = \left(\begin{bmatrix}
            2 & -1 & 0\\
            -1 & 1 & 1
        \end{bmatrix} \cdot \begin{bmatrix}
            2 & -1 \\ -1 & 1 \\
            0 & 1
        \end{bmatrix}\right)^{-1}\cdot \begin{bmatrix}
            2 & -1 & 0\\
            -1 & 1 & 0
        \end{bmatrix} = \begin{bmatrix}
            5 & -3 \\
            -3 & 3
        \end{bmatrix}^{-1} \begin{bmatrix}
            2 & -1 & 0\\
            -1 & 1 & 0
        \end{bmatrix} = \\ = \dfrac{1}{6} \begin{bmatrix}
            3 & 3 \\
            3 & 5
        \end{bmatrix} \cdot \begin{bmatrix}
            2 & -1 & 0\\
            -1 & 1 & 0
        \end{bmatrix} = \dfrac{1}{6} \begin{bmatrix}
            3 & 0 & 0\\
            1 & 2 & 0
        \end{bmatrix}
    \end{gather*}
    All things considered, we can obtain pseudoinverse matrix $A^+$:
    \[
        \dfrac{1}{36} \begin{bmatrix}
            5 & -2 \\
            -2 & 2\\
            1 & 2
        \end{bmatrix} \cdot \begin{bmatrix}
            3 & 0 & 0\\
            1 & 2 & 0
        \end{bmatrix} = \dfrac{1}{36} \begin{bmatrix}
            13 & -4 & 0\\
            -4 & 4 & 0\\
            5 & 4 & 0
        \end{bmatrix}
    \]
\end{solution}