\section{Статистическая проверка гипотез и ее приложения}

\begin{definition}{(Формальное определение статистической гипотезы)}{}
    Статистическая гипотеза -- утверждение, относительно распределения и свойств случайной величины, которое можно опровергнуть или не опровергать на основе выборки.
\end{definition}

\par Статистическая гипотеза позволяет проверять утверждения относительно всей генеральной совокупности, располагая лишь случайной выборкой.

\begin{note}{}{}
    Задача проверки статистической гипотезы формируется в виде двух гипотез:
    \begin{itemize}
        \item $H_0: $ -- нулевая гипотеза;
        \item $H_1: $ -- альтенативная гипотеза.
    \end{itemize}
\end{note}

\subsection*{Этапы проверки статистической гипотезы и ее возможные результаты}
\par
\begin{enumerate}
    \item Необходимо сформулировать утверждение, которое необходимо проверить -- гипотезу $H_0$ и альтернативную гипотезу $H_1$;
    \item Задать уровень значимости $\alpha$;
    \item Рассчитать статистику критерия $\phi$, которая зависит от результатов наблюдений и является случайной величиной;
    \item Построить критическую область, выделяя из области значений $\phi$ множество $C$ таким образом, что выполняется условие:
    \[
        P(\phi \in C) = \alpha.    
    \]
    \item сделать вывод о ложности проверяемого утверждения на основе попадания или непопадания значения $\phi$, рассчитанное на основеп выборки, в критическую область критерия.
\end{enumerate}

\par 
Гипотезы бывают односторонними и двусторонними:
\begin{enumerate}
    \item Односторонняя гипотеза:
    \[
        \begin{array}{c}
            H_0 : \ \theta = \theta_0,\\
            H_1 : \ \theta < \theta_0
        \end{array}  
    \]
    \item Двусторонняя гипотеза:
    \[
        \begin{array}{c}
            H_0: \ \theta = \theta_0\\
            H_1: \ \theta \neq \theta_0.
        \end{array}  
    \]
\end{enumerate}

\par 
Проверка гипотезы может иметь два результата:
\begin{enumerate}
    \item Отвергнуть гипотезу $H_0$ и принять $H_1$;
    \item Не отвергать гипотезу $H_0$.
\end{enumerate}

\begin{note}{}{}
    Невозможность отвергнуть гипотезу $H_0$ не означает, что она верна.
\end{note}

\subsection*{Ошибки первого и второго рода}
% ! to do
\subsection*{Возможные результаты}

\par 
\Ex Проверка гипотезы о генеральном среднем. 
\[
    H_0: \ \mu = \mu_0  
\]

\begin{enumerate}
    \item В случае, когда $n > 30, \ \sigma$ известна:
    \[
        Z = \dfrac{\overline{X} - \mu_0}{\dfrac{\sigma}{\sqrt{n}}} \sim \mathcal{N}(0;1).  
    \]
    \item В случае, когда $n > 30, \ \sigma$ неизвестна:
    \[
        Z = \dfrac{\overline{X} - \mu_0}{\dfrac{S}{\sqrt{n}}} \sim \mathcal{N}(0;1),  
    \]
    где $S$ -- точечная оценка дисперсии.
    \item В случае $n < 30, \ \sigma$ известна:
    \[
        Z = \dfrac{\overline{X} - \mu_0}{\dfrac{\sigma}{\sqrt{n}}} \sim \mathcal{N}(0; 1)
    \]
    \item $n < 30, \ \sigma$ неизвестна:
    \[
        T = \dfrac{\overline{X} - \mu_0}{\dfrac{S}{\sqrt{n}}} \sim St(n-1)  
    \]
\end{enumerate}

\par Проверка гипотезы о генеральной совокупности:
\[
    H_0 : \sigma^2 = \sigma_0^2.  
\]
Используется критерий:

\[
    \Xi^2 = \dfrac{(n-1)s^2}{\sigma^2} \sim \Xi^2(n-1).  
\]

\par Проверка гипотезы о генеральной доле:

\[
    H_0: \ p = p_0.  
\]

Используется критерий:
\[
    z = \dfrac{p-p_0}{\sqrt{\left(\dfrac{p_0(1-p_0)}{n}\right)}} \sim \mathcal{N}(0;1).
\]

\begin{definition}{(Мощность теста)}{}
    Мощностью теста называвется вероятность отвержения $H_0$ при условии, что альтернативная гипотеза верна:
    \[
        \text{Power} = P(\text{отвергнуть} H_0| \ H_1 \ \text{ верна})  = 1- \beta. 
    \]
\end{definition}

\begin{note}{}{}
    С ростом мощности теста вероятность совершить ошибку второго рода сокращается.
\end{note}

\par Гипотезы бывают простыми и сложными:
\begin{itemize}
    \item Простая гипотеза:
    \[
        H_0: \ \theta = \theta_0;  
    \]
    \item Сложная гипотеза:
    \[
        \begin{array}{c}
            H_0: \ \theta \neq \theta_0;  \\
            H_0: \ \theta > \theta_0;\\
            H_0: \ \theta < \theta_0            
        \end{array}
    \]
\end{itemize}