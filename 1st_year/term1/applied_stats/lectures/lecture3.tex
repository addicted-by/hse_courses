\section{Оценивание параметров в практике статистического анализа}
\par
В статистике наблюдения $x = \left(x_1, \ldots, x_n\right)$ рассматриваются как реализация случайного вектора $X_n = \left(X_1, \ldots, X_n\right)$, который имеет определенный закон распреления. Задача статистического оценивания заключается в оценке характеристик неизвестного распределения случайного вектора $X_n$, используя его реализацию $x$.

\begin{definition}{(Оценка)}{}
    Оценка $\hat{\theta}$ параметра $\theta$ -- некоторая функция от наблюдений, принимающая значение параметра, которое на практике используют вместо неизвестного значения параметра $\theta$.
\end{definition}

\par
Существует два подхода к статистическому оцениванию параметров:
\begin{enumerate}
    \item[1) ] $\theta$ -- неслучайная величина, неизвестная;
    \item[2) ] $\theta$ -- случайная величин с известной плотностью распределения $f(\theta)$ (Байесовский подход) 
\end{enumerate}

\begin{theorema}{}{}
    Байесовский критерий минимального среднего риска оценивания:
    \[
        \begin{array}{c}
           \displaystyle \overline{r} = \int\limits_{\theta}\int\limits_{\hat{\theta}} r\left(\theta; \hat{\theta}\right)f(\theta; x) dxd\theta\\[0.2cm]
           \displaystyle f(\theta; x) = f(\theta)f(x| \theta)\\[0.5cm]
           \displaystyle\overline{r} = \int\limits_{\theta}\int \limits_{-\infty}^\infty r\left(\theta; \hat{\theta}\right)f(\theta)f(x|\theta) dxd\theta,
        \end{array}  
    \]
    где
    \[
        \int\limits_{-\infty}^\infty r \left(\theta; \hat{\theta}\right) f(\theta) f(x|\theta) dx \to \min
    \]
    условный средний риск.
\end{theorema}

\par
Два вида представления оценок:
\begin{itemize}
    \item Точечный $\hat{\theta}$
    \item Интервальный $P\left(\theta_{\min} \leq \theta \leq \theta_{\max}\right) = r$
\end{itemize}

\subsection*{Точечное оценивание параметров}

\begin{note}{}{}
    Свойства оценок:
    \begin{itemize}
        \item Несмещенность оценки -- математическое ожидание оценки совпадает с оцениваемым параметром:
        \[
            \E\hat{\theta} = \theta.  
        \]
        \item Состоятельность --  при увеличении числа наблюдений оценка сходится по вероятности к оцениваемому параметру;
        \item Эффективность -- несмещенная ценка, дисперсия которой является минимальной по сравнению с другими оценками
    \end{itemize}
\end{note}

\par
Метод моментов: выборочные моменты используются в качестве оценок моментов генеральной совокупности.

\par 
Рассмотрим непрерывную случайную величину $X$, которая в результате $n$ испытаний принимает значения $x_1, \ldots, x_n$. Нам известен вид функции плотноти $f(x; \theta)$ с неизвестным параметром $\theta$. Для нахождения оценок ставят оптимизационную задачу максимизации функции правдоподобия вида:
\[
    L\left(x_1, \ldots, x_n; \ \theta\right) = f\left(x_1;  \theta\right)\ldots f\left(x_n; \theta\right) 
\]
Для работы с производными рассматривают логарифм функции правдоподобия.

\Ex Exponential distribution:
\[
    f(x) = \left\{
        \begin{array}{c}
        \displaystyle\lambda e^{-\lambda x}, \ x > 0\\
        \displaystyle 0, x \leq 0
    \end{array}
    \right.  
\]
\par
Likelihood:
\[
    L(x; \lambda) = \prod\limits_{i=1}^n \lambda e^{-\lambda x}   
\]
\par
Log-likelihood:
\[
    \ln L\left(x_1, \ldots, x_n, \lambda\right) = n\ln \lambda - \lambda \sum\limits_{i=1}^n x_i  
\]
\par
The result of partial derivation:
\[
    \dfrac{d \ln L}{d\lambda} = \dfrac{n}{\lambda} - \sum \limits_{i=1}^n x_i = 0  
\]
\par
Now we can obtain estimated value:
\[
    \lambda^* = \dfrac{n}{\sum\limits_{i=1}^n x_i}.
\]
\subsection*{Примеры точечных оценок наиболее важных характеристик}
\begin{itemize}
    \item Оценка математического ожидани -- выборочное среднее:
    \[
        \E X \approx \overline{x} = \dfrac{1}{n}\sum\limits_{i=1}^n x_i  
    \]
    \item Оценка генеральной дисперсии -- выборочная дисперсия:
    \[
        \var X \approx S^2 = \dfrac{1}{n} \sum\limits_{i=1}^n \left(x_i - \overline{x}\right)^2.  
    \]
    \item Несмещенная оценка дисперсии:
    \[
        \var X \approx S^2 = \dfrac{1}{n-1} \sum\limits_{i=1}^n \left(x_i - \overline{x}\right)^2.  
    \]
    \item Выборочная оценка генеральной доли:
    \[
        p = \dfrac{m}{n}.  
    \]
\end{itemize}
\par 
Для обеспечения робастности используется метод максимального квази-правдоподобия
\subsection*{Интервальное оценивание параметров}
\par 
Пусть имеется выборка $x = \left(x_1, \ldots, x_n\right)$ из генеральной совокупности $X = \left(X_1, \ldots, X_n\right)$. Целью является построения доверительного интервала $\left(\theta_{\min}; \theta_{max}\right)$, такого что:
\[
    P\left(\theta_{\min} < \theta_{\max}\right) = \gamma,  
\]
где $\gamma$ -- надежность.
\par
Этапы построения интервальных оценок:
\begin{itemize}
    \item Выбор $\gamma$;
    \item $\zeta = \mu(\theta)$, причем подразумевается,что $P(\zeta)$ известно;
    \item Построение доверительного интервала для статистики:
    \[
        P\left(\zeta_{\min} < \zeta < \zeta_{\max}\right) = \gamma.  
    \]
    \item Построение доверительного интервала для оценки:
    \[
        P\left(\theta_{\min} < \theta < \theta_{\max}\right) = \gamma.
    \]
\end{itemize}

\Ex (для построения доверительного интервала для среднего)
Пусть имеется выборка $x = \left(x_1, x_2, \ldots, x_n\right)$, кроме того $x_i \sim N(a, \sigma^2)$ i.i.d и $\sigma^2$ известно.
\begin{itemize}
    \item Выберем $\overline{X}$ в качестве статистики.
    \[
        \E(\overline{X}) = a \hspace*{0.5cm} \var \overline{X} = \dfrac{1}{n}\sigma^2  
    \]
    $\overline{X} \sim N\left(a, \dfrac{\sigma^2}{n}\right)$:
    \[
        \begin{array}{c}
            \displaystyle \overline{X} - \dfrac{\sigma}{\sqrt{n}} < a < \overline{X} + \dfrac{\sigma}{\sqrt{n}} \\[0.5cm]
            \displaystyle P\left(-t_{\gamma} < \dfrac{\overline{X} - a}{\sigma} - \sqrt{n} < t_{\gamma}\right) = \gamma\\[0.5cm]
            \displaystyle P\left(\overline{X} - t_{\gamma} \dfrac{\sigma}{\sqrt{n}} < a < \overline{X} + t_{\gamma} \dfrac{\sigma}{\sqrt{n}}\right) = \gamma.
        \end{array}  
    \]
\end{itemize}