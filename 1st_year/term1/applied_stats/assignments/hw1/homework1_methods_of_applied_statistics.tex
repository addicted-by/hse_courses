\documentclass[12pt]{report}
\usepackage{../mystyle}
\begin{document}
\pagestyle{fancy}
% \boldmath
\fancyhead[R]{Рябыкин Алексей}
\fancyhead[L]{Методы прикладной статистики}
\fancyfoot[C]{--\thepage--}
\fancyfoot[L]{2022}
\fancyfoot[R]{v1.0}
\begin{center}
    {\textbf {\huge Домашнее задание 1}}\\[5mm]
{\large Рябыкин Алексей} \\[5mm]
\today\\[5mm]
\end{center}
Задание 1. Определить закон распределения случайной величины, являющейся частотой выпадения орла при трех подбрасываниях монеты. Найти математическое ожидание, дисперсию, моду, медиану, квантили $0.1$, $0.25$, $0.9$, коэффициенты ассиметрии и эксцесса.

\begin{solution}
    Запишем закон распределения:
    \begin{center}
        \begin{tabular}{|c|c|c|c|c|}
            \hline
            Событие $X$ & орел не выпал  & орел выпал 1 раз & орел выпал 2 раза & орел выпал 3 раза \\ \hline\xrowht{15pt}
            $x_i$       & $0$            & $1$              & $2$               & $3$               \\ \hline\xrowht{30pt}
            $P(x_i)$    & $\dfrac{1}{8}$ & $\dfrac{3}{8}$   & $\dfrac{3}{8}$    & $\dfrac{1}{8}$    \\ \hline
            \end{tabular}
    \end{center}
    
    Найдем математическое ожидание случайной величины:
    \[
        \E X = \sum\limits_{i=0}^3 x_ip_i = 0 \cdot \dfrac{1}{8} + 1 \cdot \dfrac{3}{8} + 2\cdot \dfrac{3}{8} + 3\cdot \dfrac{1}{8} = \dfrac{3}{2}
    \]  
    
    Найдем дисперсию дискретной случайной величины:
    \[
       \var X = \E [X^2] - [\E X]^2 = 3 - \dfrac{9}{4} = \dfrac{3}{4}
    \]
    
    Мода: $x_i=1$ или $x_i=2$.
    \par
    Медиана: запишем кумулятивную функцию распределения
    \begin{center}
        \begin{tabular}{|l|l|l|l|l|}
            \hline
            $P(X \leq x_i)$ & $0.125$ & $0.5$ & $0.625$ & $1$ \\ \hline
        \end{tabular}
    \end{center}
    \par
    Медиану можно определить как среднеарифметическое наибольшего значения $f_1$, при котором выполняется $P(X \leq f_1) \leq 0.5$ и наименьшего $f_2$, для которого $P(X \leq f_2) \geq 0.5$. $f_1 = 1$ и $f_2 = 2$. Тогда медиана:
    \[
        Me = \dfrac{f_1 + f_2}{2} =  1.5 
    \]
    \par 
    Квантиль 0.1: $q$-квантиль в дискретном случае можно определить как любое число $u_q(F)$, лежащее между двумя возможными соседними значениями $x_{i(q)}$ и $x_{i(q)+1}$, такое, что
    \[
       F(x_{i(q)}) < q \ \text{ и } F(x_{i(q) + 1}) \geq q.  
    \]
    \par
    Тогда квантиль $0.1$ -- любое число из диапазона $(-\infty, 0)$.
    \par 
    Квантиль 0.25: $(0, 1)$
    \par
    Квантиль 0.9: $(2,3)$
    \par
    Коэффициент ассиметрии: вычисляется по формуле
    \[
        \beta_1 = \dfrac{\mu_3}{\sigma^3},  
    \]
    где $\mu_3$ -- центральный момент третьего порядка, равный 
    \[
        \mu_3 = v_3 - 3v_1v_2 + 2v_1^3,
    \]
    где $v_i = \E [X^i]$ -- начальный момент $i$-го порядка, и $\sigma$ -- средне-квадратичное отклонение.
    Найдем начальные и центральные моменты:
    \begin{gather*}
        v_1 = \dfrac{3}{2}, \ v_2 = 3, \ v_3 = \dfrac{54}{8}, v_4 = \dfrac{132}{8}\\
        \mu_3 = \dfrac{54}{8} - 3 \dfrac{3}{2} 3 + 2\dfrac{27}{8} = 0 \Rightarrow \beta_1 = 0.\\
        \mu_4 = \dfrac{21}{16}
    \end{gather*} 
    \par 
    Коэффициент эксцесса:
    \[
        \beta_2 = \dfrac{\mu_4}{\sigma^4} - 3 \approx -0.67
    \]
\end{solution}
\par
Задание 2.1 Случайная величина задается следующим законом распределения:
\[
    f(x) = \left\{
        \begin{array}{c}
            0, \ x < 5,\\
            0.2x, \ 0 \leq x \leq 5,\\
            1, \ x > 5.
        \end{array}
     \right.  
\]
Найти математическое ожидание, дисперсию, моду, медиану, квантили $0.1$, $0.25$, $0.9$, коэффициенты ассиметрии и эксцесса.
\begin{solution}
    Заметим, что это равномерное распределение c $b = 5, \ a = 0$. Тогда, математическое ожидание:
    \[
        \E = \dfrac{a+b}{2} = \dfrac{5}{2},
    \] 
    дисперсия:
    \[
        \var = \dfrac{(b-a)^2}{12} = \dfrac{25}{12}
    \]
    коэффициенты ассиметрии и эксцесса:
    \[
        \beta_1 = 0 \hspace*{0.5cm}  \beta_2 = -\dfrac{6}{5},
    \]
    мода $[0,5]$, медиана $2.5$. 
    \par
    Квантили:
    \begin{gather*}
        0.2x_{0.1} = 0.1 \Rightarrow x_{0.1} = 0.5, \\
        0.2x_{0.25} = 0.25 \Rightarrow x_{0.25} = 1.25, \\
        0.2x_{0.9} = 0.9 \Rightarrow x_{0.9} = 4.5.
    \end{gather*}
\end{solution}
\end{document}
